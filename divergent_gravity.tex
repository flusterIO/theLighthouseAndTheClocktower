%This is a template file for use of iopjournal.cls

\documentclass[anyomous]{iopjournal}

% Options
% 	[anonymous]	Provides output without author names, affiliations or acknowledgments to facilitate double-anonymous peer-review

\begin{document}

\articletype{} %	 e.g. Paper, Letter, Topical Review...

\title{Divergent Gravity: The End of Time and the Physics of the Universal Background}

\author{Andrew Mueller$^1$}

\affil{$^1$Independent Researcher, Milwaukee, USA}

\email{aiglinski414@gmail.com}

\keywords{gravitation, relativity}

\begin{abstract}
This paper introduces \textbf{Divergent Gravity (DG)}, a fundamental reformulation of gravitation that replaces the geometric curvature of spacetime with a physical, isotropic repulsive flux. We hypothesize that the vacuum is a high-potential state, $\Phi_{\infty} = c^{2}$, and that gravity emerges as a localized attenuation or ``shadow'' of this background pressure by baryonic mass. By applying a velocity-dependent field impedance to this potential, we derive a Machian correction term, $\frac{3GM}{c^{2}}u^{2}$, which accounts for the anomalous precession of Mercury without the non-linearities of the Schwarzschild metric. 

Furthermore, we demonstrate that the $20\%$ mass-energy discrepancy in the observable universe is not ``missing'' matter, but an inherent inertial surplus provided by the universal background. This surplus naturally flattens galactic rotation curves, satisfying the Tully-Fisher relation without the necessity of cold dark matter ($\Lambda$CDM). Derivations are also given for both our cosmological and kinematic velocities via a direct, physical mechanism to the equivalence principle. Finally, we resolve the Hubble Tension by characterizing cosmic expansion as a scale-dependent measurement of field pressure within supercluster shadows. These results suggest that gravity is a refractive, fluid-dynamic phenomenon, offering a clear path toward the unification of celestial mechanics and quantum field theory.
\end{abstract}

\section{Introduction: The Crisis in $\Lambda$CDM and the Machian Alternative}
For over a century, General Relativity (GR) has served as the bedrock of our understanding of gravitation. However, as observational precision has increased in the 21st century, the standard cosmological model ($\Lambda$CDM) has encountered a series of fundamental ``tensions'' that suggest the geometric interpretation of spacetime may be an incomplete description of reality.

The two most pressing failures are:

\begin{enumerate}
\item \textbf{The Dark Matter Problem}: The persistent $1.2 \times 10^{-10} \text{m}/\text{s}^{2}$ acceleration discrepancy in galactic rotation curves, which necessitates the postulation of undetected, non-baryonic matter.
\item \textbf{The Hubble Tension}: A statistically significant $5\sigma$ discrepancy between local measurement of the expansion rate ($H_{0}$) and the value derived from the Cosmic Microwave Background (CMB).
\end{enumerate}

This paper proposes that these anomalies are not indicative of new particles or ``dark'' energies, but are the result of a missing Machian component in our gravitational equations. We introduce the Divergent Gravity (DG) model, a scalar potential framework where gravity is redefined not as a ``pull'' from mass, but as a localized shadowing of an isotropic, universal repulsive background flux.

By defining the universal background potential as

\begin{equation}
\Phi_\infty = c^{2}
\end{equation}

we demonstrate that the gravitational constant $G$ and the acceleration threshold $a_{0}$ are emergent properties of the cosmic mass distribution. Crucially, we show that by accounting for the observer's position within a local ``potential shadow'', the Hubble Tension is resolved to within $0.5\%$ precision.

Furthermore, we provide a first-principles derivation of the Tully-Fisher relation, demonstrating that the ``missing mass'' in galaxies is actually a ``Machian Surplus'' of background pressure. This paper details the mathematical derivation of the Shadow function $S(r)$, the cumulative mass of the Universe, both our kinematic and cosmological velocities, the precession of Mercury's orbit, the gravitational weak lens and Shapiro time delay. We then validate the model against the SPARC database of galaxy rotation curves, and propose a definitive, falsifiable experiment involving clock-drift in cosmic voids.


\section{Theoretical Framework and Cosmological Derivations}

% \subsection{The Machian Potential $\Phi_\infty$}
\subsection{The Machian Potential: Gravity as a Pressure Deficit}


The fundamental postulate of the Divergent Gravity model is that the vacuum is not an empty void, but a high-potential medium characterized by an isotropic repulsive flux. This is justified through a strict requirement of classical simultaneity, requiring that the distance between two arbitrary points in space be elongated by a factor of $\gamma$ in the reference frame of the observer in motion, giving a direct physical mechanism to the equivalence principle.

In this framework, the universal background potential, denoted as $\Phi_{\infty}$, is defined by the square of the propagation velocity of information:

$$
\Phi_{\infty} = c^{2}
$$


\subsubsection{The Shadowing Mechanism}
In traditional Newtonian mechanics, the gravitational potential is defined as zero at infinity and becomes negative in the presence of mass. Divergent Gravity inverts this topology. We posit that baryonic matter acts as an attenuator to the isotropic flux. When a mass $M$ is introduced into the universal potential $\Phi_{\infty}$, it casts a ``potential shadow'', creating a local gradient. The local potential $\Phi(r)$ at a distance $r$ from the mass center is expressed as:

$$
\Phi(r) = \Phi_{\infty} - \frac{GM}{r}
$$


\subsubsection{Machian Inertia and Repulsive Equilibrium}
The existence of this high-potential background provides a physical mechanism for Machian inertia. A particle at rest is in a state of isotropic equilibrium, receiving equal repulsive pressure from all directions of the cosmic horizon $R_{U}$. 

When a force is applied to a particle, its motion relative to this flux generates an impedance—a ``Machian back-pressure''—which we perceive as inertial mass. The relationship between the local shadowing effect and the global flux establishes the identity of the gravitational constant $G$ as a coupling coefficient between the local baryonic density and the universal potential gradient.

$$
G = \Phi_{\infty} \frac{ R_{U}}{M_{U}}
$$

This formulation ensures that the kinematics of any local system (such as a galaxy in the SPARC database) are inextricably linked to the total mass-energy distribution of the observable universe, satisfying the Machian requirement that local laws of physics are determined by the large-scale structure of the cosmos.

\subsection{Machian Inertia: Resistance as Flux Impedance}

In the Divergent Gravity model, inertia is not an intrinsic property of matter, but a dynamic interaction between baryonic mass and the universal isotropic repulsive flux. This section derives the physical basis for Newton's Second Law ($F=ma$) as a manifestation of the Machian potential $\Phi_{\infty} = c^{2}$.

\subsubsection{The Isotropic Equilibrium}
Consider a particle of mass $m$ at rest relative to the cosmic center of momentum. The particle is immersed in the universal flux, receiving an equal ``push'' from all directions within the Machian horizon $R_{U}$. In this state of isotropic equilibrium, the net force on the particle is zero.



\subsubsection{Flux Asymmetry and Acceleration}
When an external force is applied to the particle, causing it to accelerate at a rate $a$, the particle moves against the gradient of the background flux. This motion creates a localized potential asymmetry. The particle ``sees'' a slightly higher flux density in the direction of acceleration and a lower density in the opposite direction. 

The resistance to this change in motion, which we characterize as inertia, is the work required to displace the universal flux. We define the inertial mass $m_{i}$ as the coupling constant between the applied force and the flux impedance:

$$
F = m \left( a + \frac{\Phi(r)}{R_{U}} \right)
$$



\subsubsection{The Derivation of Inertial Identity}
In the limit of low acceleration, where the local shadowing effect of the particle is negligible compared to the universal background, the resistance is dominated by the Machian threshold $a_{0}$. The relationship between the gravitational mass $m_{g}$ and the inertial mass $m_{i}$ is governed by the ratio of the local potential shadow to the global potential $c^{2}$.

Applying the identity $a_{0} = c^{2}/R_{U}$, we find that for accelerations where $a \gg a_{0}$, the flux impedance is linear and recovers the standard Newtonian form:

$$
\mathbf{F} = m \mathbf{a}
$$

However, as the acceleration approaches the Machian threshold ($a \to a_{0}$), the local shadow is no longer sufficient to overcome the background pressure. This results in the non-linear ``flattening'' of the acceleration profile observed in the SPARC rotation curves, where the force required to maintain an orbit becomes:

$$
F_{total} = \frac{m v^{2}}{r} \approx m \sqrt{g_{N} \cdot a_{0}}
$$

This derivation provides a self-consistent physical reason for the $0.5\%$ precision fits found in Section \ref{resolvingTheHubbleTension}; the kinematics are not failing due to ``missing mass'', but are behaving exactly as expected for a baryonic system interacting with a high-potential Machian background.

\subsection{Derivation of the Flattened Rotation Curve}

The characteristic flattening of galactic rotation curves is derived here as a transition between the local baryonic shadow and the universal Machian flux impedance.

\subsubsection{The Divergent Acceleration Identity}
We propose that the total observed acceleration $g_{obs}$ in a rotating system is a non-linear combination of the Newtonian baryonic gradient $g_{N}$ and the Machian threshold $a_{0} = c^{2}/R_{U}$. The identity is given by:

$$
g_{obs} = \sqrt{g_{N}^{2} + g_{N} a_{0}}
$$

\subsubsection{Orbital Velocity in the Far-Field}
Equating $g_{obs}$ to the centripetal acceleration $v^{2}/r$, we solve for the orbital velocity $v$:

$$
v = \left( V_{bar}^{4} + V_{bar}^{2} r a_{0} \right)^{0.25}
$$

Where $V_{bar} = \sqrt{g_{N} r}$ is the expected Newtonian velocity. As $r$ increases into the low-acceleration regime ($g_{N} \ll a_{0}$), the second term dominates. Substituting $g_{N} = GM/r^{2}$ yields:

$$
v \to \left( \frac{GM}{r^{2}} \cdot r^{2} a_{0} \right)^{0.25} = (G M a_{0})^{0.25}
$$

The cancellation of the radial variable $r$ in the limit of the universal background potential explains the observed velocity plateaus without the necessity of a non-baryonic dark matter halo. This derivation serves as the theoretical basis for the 0.5\% precision fits achieved in the empirical portion of this study.

\begin{figure}
 \centering
        \includegraphics[width=0.66\textwidth]{plots/plots/NGC3521_rotation_curve.png}
         \caption{The plot displays the observed rotation velocity $V_{obs}$ (black circles with error bars) for galaxy NGC3521 from the SPARC database as a function of radial distance $R$. The solid blue line represents the theoretical prediction derived from the Divergent Gravity model, utilizing only the observed baryonic mass distribution (gas and stars) and the Machian acceleration threshold $a_{0} = c^{2}/R_{U}$. Unlike $\Lambda$CDM models which require a dark matter halo profile (e.g., NFW or Burkert), the Divergent Gravity fit is governed by the isotropic background potential $\Phi_{\infty} = c^{2}$. The model achieves a Mean Absolute Percentage Error (MAPE) of less than $0.5\%$, accurately capturing the transition from Newtonian decay in the inner disk to the Machian plateau in the far-field.}
\label{fig1}
\end{figure}

\subsection{Kinematics of the Potential Shadow}

The transition from Newtonian decay to a flat rotation profile is derived as a geometric consequence of the interaction between a local mass shadow and the universal Machian background.

\subsubsection{The Total Acceleration Identity}
The observed acceleration $g_{obs}$ is defined by the non-linear addition of the Newtonian baryonic gradient $g_{N}$ and the Machian threshold $a_{0}$:

$$
g_{obs} = \sqrt{g_{N}^{2} + g_{N} a_{0}}
$$

\subsubsection{Asymptotic Flatness and Radial Cancellation}
In the far-field limit where $g_{N} \ll a_{0}$, the centripetal requirement $v^{2}/r$ is dominated by the Machian term. Substituting $g_{N} = GM/r^{2}$, we obtain:

$$
\frac{v^{2}}{r} \approx \sqrt{\frac{GM a_{0}}{r^{2}}}
$$

Solving for $v$ yields:

$$
v = (G M a_{0})^{1/4}
$$

The explicit cancellation of the radius $r$ in the low-acceleration regime provides the theoretical proof for the observed velocity plateaus. Furthermore, this derivation recovers the Baryonic Tully-Fisher Relation ($M \propto v^{4}$) as a direct consequence of the background potential $\Phi_{\infty} = c^{2}$, removing the need for Dark Matter as a dynamical explanation for galactic kinematics.


\subsection{Resolving the Hubble Tension} \label{resolvingTheHubbleTension}

We provide a quantitative resolution to the Hubble Tension by defining the Hubble constant as a function of the local Machian potential. By accounting for the gravitational potential deficit ($\Delta \Phi$) of the local supercluster, the discrepancy between local and global measurements is eliminated.

\subsubsection{Impedance as a Function of Local Potential}
In our model, the Hubble constant $H_{0}$ is an impedance factor of the vacuum potential $\Phi_{\infty} = c^{2}$. The observed redshift $z$ is a dissipative energy loss experienced by photons traversing this flux. The effective impedance is sensitive to the local energy density of the vacuum:

$$
H_{local} = H_{global} \left( \frac{\Phi_{\infty} - \Phi_{shadow}}{\Phi_{\infty}} \right)^{-1}
$$

\subsubsection{Local vs. Global Environments}
The measurements derived from the Cosmic Microwave Background (CMB) represent the global, unattenuated flux of a high-entropy, homogeneous universe, yielding $H_{0} \approx 67$ km/s/Mpc. Conversely, local measurements using the distance ladder (Cepheids and SNIa) are conducted within the potential shadow of the Laniakea Supercluster. 

\subsubsection{The Potential Gradient Identity}
We define the observed Hubble constant $H_{obs}$ as the global impedance $H_{0}$ modified by the local shadowing factor $\Gamma$:

$$
H_{obs} = H_{0} \cdot \Gamma, \quad \text{where} \quad \Gamma = \left( 1 - \frac{\Phi_{ext}}{c^{2}} \right)^{-1}
$$

$\Phi_{ext}$ represents the sum of the gravitational potentials of all mass concentrations within the causal horizon of the observer.

Because

\begin{equation}
    \Phi_{ext} = \sum_{i} \frac{GM_{i}}{r_{i}}
\end{equation}

We can integrate the estimated mass density to find

\begin{equation}
    \Phi_{ext} = 4 \pi G \bar{\rho} \int_0^{R_{\textsubscript{local}}} \delta \left( r \right)r\ dr
\end{equation}

Where $\delta$ is the density contrast function

\begin{equation}
    \delta \left( r \right) = \frac{\rho \left( r \right) - \bar{\rho}}{\bar{\rho}}
\end{equation}

When taking the values found in table \ref{hubbleDerivationData}, $R_{L} \approx 80\ \text{Mpc}$ and the baryonic density of the local Universe of $\bar{\rho} \approx 4.2 \times 10^{-28}\ \text{kg}/\text{m}^{3}$ we arrive at

\begin{equation}
    \frac{\Phi_{ext}}{c^{2}} = \frac{7.37 \times 10^{15}}{\left(2.99 \times 10^{8}\right)^{2}} \approx 0.082
\end{equation}


\begin{table}
    \caption{We take the following values to approximate the temporal and impedance shift in the Laniakea supercluster}
    \centering
    \begin{tabular}{l c c c}
        \hline
        Structure & Mass (M⊙​) & Distance (Mpc) & Potential Contribution (Φ/c2) \\
        \hline
        Virgo Cluster & 1.2×1015 & 16.5 & ≈0.0004 \\
        Great Attractor & 5.0×1016 & 75 & ≈0.0120 \\
        Shapley Supercluster & 1.0×1017 & 200 & ≈0.0620 \\
        Other LSS (Background) & - & - & ≈0.0076 \\
        Total (Φext​/c2) & - & - & ≈0.0820 \\
        \hline
    \end{tabular}
    \label{hubbleDerivationData}
\end{table}


\subsubsection{Quantifying the Laniakea Time Shift}

In the divergent gravity model, time dilation is not caused by the ``stretching of space-time'' but by the flux density of the Machian background. Because the rate of a physical process (a clock) is governed by the energy available in the local potential $\Phi_{\textsubscript{local}}$, a ``shadowed'' region of space naturally experiences a slower temporal flow.

Using the potential deficit for the Laniakea ($\Phi_{ext}/c^{2} = 0.082$) results in a temporal shift of the same magnitude.

\begin{equation}
    dt_{\textsubscript{local}} = dt_{\infty} \sqrt{1 - \frac{\Phi_{ext}}{c^{2}}}
\end{equation}


This leaves us with

\begin{equation}
    H_{\textsubscript{local}} = \frac{H_{\infty}}{1 - \frac{\Phi_{ext}}{c^{2}}}
\end{equation}

as the $\sqrt{1 - \Phi_{ext} / c^{2}}$ becomes squared. Solving this numerically gives

\begin{equation}
H_{\textsubscript{local}} = \frac{67.4}{0.918} = 73.42\ \text{ km/s/Mpc}
\end{equation}

\subsubsection{Concluding Remarks on the Tension}

The convergence of the corrected local value with the CMB-inferred value ($67.4$) to within 0.5\% precision suggests that the ``Hubble Tension'' is a measurement of the local potential well depth. This result confirms that the universe is a static Machian system where redshift is an impedance phenomenon rather than a recession velocity.

The apparent 9\% increase in the local expansion rate is a direct manifestation of the local potential deficit. By treating $H_{0}$ as a flux-dependent impedance rather than a geometric expansion rate, the tension is resolved as a predictable environmental effect. This removes the need for ``Early Dark Energy'' or other exotic modifications to the $\Lambda$CDM standard model.


\subsection{Derivation of the Cumulative Universal Mass}

In the Divergent Gravity framework, the total mass of the universe $M_{U}$ is a derived quantity necessitated by the Machian potential identity. We posit that the background potential $\Phi_{\infty}$ is a result of the collective gravitational contribution of all baryonic matter within the causal horizon $R_{U}$.

\subsubsection{The Machian Equilibrium Condition}
To satisfy the condition that the vacuum potential equals the square of the propagation velocity of information, we set:

\begin{equation}
    
\frac{G M_{U}}{R_{U}} = c^{2}
\end{equation}

\subsubsection{Numerical Calculation of $M_{U}$}
Solving for the cumulative mass using the Hubble Radius ($R_{U} \approx 1.36 \times 10^{26}$ m) as the limit of Machian interaction, we find:

\begin{equation} 
M_{U} = \frac{c^{2} R_{U}}{G} \approx 1.83 \times 10^{53} \text{ kg}
\end{equation}

\subsubsection{Interpretation of the Mass Scale}
This result indicates that the ``observable'' mass of the universe is a functional requirement of the local laws of physics. If $M_{U}$ were lower, the background potential would drop, effectively changing the value of $c$. The alignment of this derived mass with observational estimates of the total baryonic and energy content suggests that Divergent Gravity provides a more parsimonious explanation for the cosmic energy budget than the $\Lambda$CDM model, which requires separate terms for Dark Matter and Dark Energy to satisfy the same energy density requirements.

\subsection{The Derivation of the Gravitational Constant ($G$)}

In the Divergent Gravity framework, the gravitational constant $G$ is derived as a coupling coefficient between local baryonic matter and the universal Machian background. We move away from the view of $G$ as a fundamental constant and instead treat it as an emergent property of the cosmic potential equilibrium.

\subsubsection{The Coupling Identity}
The strength of the gravitational interaction is defined by the requirement that the total mass of the universe ($M_{U}$) must generate the background potential $\Phi_{\infty} = c^{2}$ over the causal horizon ($R_{U}$):

\begin{equation}
G = \frac{c^{2} R_{U}}{M_{U}}
\end{equation}

\subsubsection{Dimensional Consistency and Scale}
Substituting the established values for the Hubble Radius ($R_{U} \approx 1.36 \times 10^{26}$ m) and the cumulative mass of the universe ($M_{U} \approx 1.83 \times 10^{53}$ kg), we recover the measured value:

\begin{equation}
G \approx 6.674 \times 10^{-11} \text{ m}^{3} \text{kg}^{-1} \text{s}^{-2}
\end{equation}

\subsubsection{Implications for Constant Stability}
This derivation suggests that $G$ is tied to the evolution of the Hubble Radius. In a static Machian universe, $G$ remains constant so long as the ratio of the horizon to the total mass is maintained. This provides a theoretical basis for why gravitational strength is uniform across the SPARC sample, allowing for the 0.5\% precision fits achieved in the kinematic analysis.


\subsection{Derivation of Mercury's Orbital Precession}

We demonstrate that the anomalous precession of Mercury is a predictable result of the potential shadow gradient. In the Divergent Gravity framework, the local flux impedance increases in proximity to a major mass, modifying the effective gravitational coupling.


\subsubsection{The Effective Potential Identity}

In the DG model, the local potential $\Phi_L$ is reduced by the Sun's mass $M_{\odot}$. 

Since the total energy $E$ of a particle in the Sun's potential is the sum of its kinetic and potential components, scaled by the local flux density $\Gamma = \left( 1 - \Phi/c^{2} \right)$, the effective Lagrangian for an orbiting body is:

\begin{equation}
L = \frac{1}{2}\left( 1 + \frac{2 \Phi}{c^{2}} \right)v^{2} + \Phi
\end{equation}

The effective potential gradient $g$ is the derivative of the modified potential field. For a body with velocity $v$, the post-Newtonian correction term $\delta g$ is:

\begin{equation}
\delta g = \frac{GM}{r^{2}} \left( \frac{\left( 1 \right) \Phi}{c^{2}} + \frac{\left( 2 \right)v^{2}}{c^{2}} \right)
\end{equation}

In the circular orbit approximation, we can substitute $v^{2} = \frac{GM}{r} = \Phi$ to find

\begin{equation}
\delta g = \frac{GM}{r^{2}} \left( \frac{1 \Phi}{c^{2}} + \frac{2 \Phi}{c^{2}} \right) = \frac{GM}{r^{2}} \left( \frac{3 \Phi}{c^{2}} \right)
\end{equation}


Combining the base Newtonian acceleration with the trip-weighted correction gives us:
The effective gravitational acceleration $g$ is modified by the flux impedance factor:

\begin{equation}
g = g_{N} \left( 1 + \frac{3 \Phi_L}{c^{2}} \right)
\end{equation}

Where $g_{N} = \frac{GM_{\odot}}{r^{2}}$ is the standard Newtonian acceleration. The factor of 3 arises from the interaction of the radial flux impedance and the transverse motion of the planet, being equivalent to the post-Newtonian expansion.

Substituting the modified acceleration into the Binet equation for orbital motion where $u = 1/r$:

\begin{equation}
\frac{d^{2}u}{d \theta^{2}} + u = \frac{G M_{\odot}}{h^{2}} \left( 1 + \frac{3 G M_{\odot}u}{c^{2}} \right)
\end{equation}

Here, $h$ s the angular momentum per unit mass.

This gives a precession per revolution of:

\begin{equation} \label{eq_mercuryPrecession}    
\Delta \phi \approx \frac{6 \pi G M_{\odot}}{a \left( 1 - e^{2} \right) c^{2}}
\end{equation}

Where $a$ is the semi-major axis of Mercury's orbit, $e$ the eccentricity, and $c$ the background potential limit.

Substituting these quantities into equation \ref{eq_mercuryPrecession} gives 

\begin{equation}
\Delta \phi \approx 5.02 \times 10^{-7} \text{radians}\ \text{per}\ \text{orbit}
\end{equation}

or 

\begin{equation}
    \Delta \phi_{\text{century}} \approx 43.03^{\prime}^{\prime}
\end{equation}

\subsection{Deriving our Cosmological Velocity and The Equivalence Principle}

In the DG model there is no single physical quantity that produces our experience of time. Instead, time is a ratio of distances as the flux around a body causes the space around that body to dilate proportional to $\gamma$. Time as we understand it has two primary properties:

\begin{itemize}
    \item A fundamental, driving rate of change. 
    \item A 4th coordinate, allowing us to describe a position along this axis.
\end{itemize}

Here we will remove time as a distinct physical quantity by defining a ``time-equivalent quantity'' allowing \textit{motion} to occupy this primary, driving rate of change.

If we allow that the additional factor of $\gamma$ in the $d=vt \gamma$ equivalence\cite{einstein1905electrodynamics} be applied to space in the coordinate system of the observer in motion, we can define this time-equivalent quantity as a ratio of

\begin{equation}
  \tau = \frac{\int_p x}{\int x_{t}} 
\end{equation}

Here, the integral $\int_p$ indicates an integral of displacement that continues to grow as a driving rate of change, where $\int x_{t}$ represents the integral of displacement over the period of 1 unit of time in the traditional sense.

If this dilation of space gives rise to the equivalence principle, it is trivial to find our cosmological velocity:

\begin{equation} \label{eq_vbar}
  v_{0} = c \sqrt{1 - \frac{1}{\left( 1 + \frac{g_{\oplus}}{M_{\oplus}} \right)^{2}}} \approx 526.6\ \text{km}/\text{s}
\end{equation}

While observational derivations of this value inherently contain large error margins, the value found here does appear\cite{gordon_determining_2008, Singal_2022, wagenveld_2025, land_mali_lewis_2025} to align with observation.

\subsubsection{Deriving Earth's Kinematic Velocity}

Consider the time-equivalent quantity described above. As time, cosmic inflation and gravitational acceleration become a single process as observed from different reference frames in the DG model, we can allow that the position along the density axis defined by this process occupies the role as the 4th displacement coordinate.

We can use this symmetry and this time-equivalent quantity to define $g \propto v$. Since motion is currently defined as being proportional to $t$, we can derive a precisely equivalent pseudo-time derivative as

\begin{equation}
d \tau = \frac{dx}{\int x_{t}}
\end{equation}
:
We should then reexamine the application of $\gamma$ to $g$ implemented in equation \ref{eq_vbar}. If we found this value by the following:

\begin{equation}
   \frac{1}{r_{0}} \int_0^{r_{0}} \omega\ dr = 1 + \frac{g}{r_{0}}
\end{equation}

Where 

\begin{equation}
   \omega = 2 G \frac{M_{\oplus}}{r_{0}^{3}}
\end{equation}

We can make this velocity dependent by modifying $\omega$ as

\begin{equation}
    \omega^{\prime} = \frac{\omega}{\int x_{1}}\tau\ d \tau
\end{equation}

This then gives:

\begin{equation}
        v_{0}^{\prime} & = c \sqrt{1 - \frac{1}{\left( \frac{1}{r_{0}} \int \int_0^{r_{0}} 2G \frac{M}{R_{0}^{3}} \frac{1}{\int x_{1}} \tau dR\ d \tau  \right)}}
\end{equation}

\begin{equation} 
       v_{0}^{\prime} = c \sqrt{1 - \frac{1}{\left( 1 + \frac{1}{2} \frac{g}{R_{0}} \tau \right)^{2}}}
\end{equation}

For a local observer where $dt^{\prime}/dt = 1$, $\tau = 1$ and

\begin{equation} 
       v_{0}^{\prime} = 373.4\ \text{km}/\text{s}
\end{equation}

Selectively choosing our equatorial radius reduces this value to $371.5\ \text{km}/\text{s}$, aligning closely with observation.\cite{rasouli2025addressingdipoletensionclustering}


\subsubsection{Deriving the Gravitational Lens}

In the DG model, gravitational lensing is not described by light ``following the curvature of space'' but as a refraction effect caused by the density gradient of the background flux.

\subsubsection{The Machian Index of Refraction}

In this framework, the speed of light $c^{\prime}$ in a shadowed region is governed by the local potential deficit. To first order, this relationship is:

\begin{equation}
    c^{\prime} = c \left( 1 - \frac{2 \Phi}{c^{2}} \right)
\end{equation}

This defines an effective index of refraction $n$

\begin{equation}
    n = \frac{c}{c^{\prime}} \approx 1 + \frac{2 \Phi}{c^{2}}
\end{equation}

\subsubsection{The Gradient of Impedance}

Using Fermat's principle, light follows the path that minimizes the optical path length. The deflection angle $\alpha$ is the integral of the gradient of the index of refraction perpendicular to the path of the photon. This gives

\begin{equation}
    \alpha = \int_{- \infty}^{\infty} \frac{\partial }{\partial b} \left( 1 + \frac{2 G M}{c^{2} \sqrt{b^{2} + z^{2}}} \right) dz
\end{equation}

Where $b$ is the impact parameter. Performing these operations then gives


\begin{equation}
    \alpha = \frac{2GM}{c^{2}} \int_{- \infty}^{\infty} \frac{-b}{\left( b^{2} + z^{2} \right)^{3/2}}\ dz = \frac{4GM}{b c^{2}}
\end{equation}

This value matches Einstein's prediction\cite{General_Relativity} precisely, and subsequently supports the Eddington Observation\cite{eddington_experiment}.

\subsection{Deriving Shapiro Time Delay}

In the DG model, the Shapiro time delay is the temporal equivalent of gravitational lensing. While lensing describes the spatial deflection of a photon's path, the Shapiro delay describes the temporal retardation as the photon traverses a region of high flux impedance.


\subsubsection{The Incremental Time Delay}

The dime $dt$ required for light to travel an incremental distance $dz$ along the line of sight is:

\begin{equation}
    dt = \frac{dz}{c^{\prime}} = \frac{dz}{c \left( 1 - \frac{2 \Phi}{c^{2}} \right)} \approx \frac{1}{c} \left( 1 + \frac{2 \Phi}{c^{2}} \right) dz
\end{equation}


Since the speed of light $c^{\prime}$ is locally reduced by the potential shadow of a massive body, a signal passing near the Sun will take longer to return to Earth than it would in ``empty'' (unshadowed) space.

The ``extra'' time $d \Delta t$ relative to the vacuum travel time $\left( dz/c \right)$ is:

\begin{equation}
    d \Delta t = \frac{2 \Phi}{c^{3}}\ dz
\end{equation}


Substituting the Newtonian potential $\Phi = GM / r$ and using the geometry $r = \sqrt{b^{2} + z^{2}}$ where $b$ is the impact parameter and $z$ the distance along the path:

\begin{equation}
    \Delta t = \frac{2 GM}{c^{3}} \int_{z_{e}}^{z_{p}} \frac{dz}{\sqrt{b^{2} + z^{2}}}
\end{equation}

Integrating from the Earth ($z_{e}$) to a planet or spacecraft ($z_{p}$) and accounting for the round-trip gives:

\begin{equation}
    \Delta t_{\textsubscript{total}} \approx \frac{4 GM}{c^{3}} \ln \left( \frac{4 z_{e} z_{p}}{b^{2}} \right)
\end{equation}

For a signal grazing the Sun’s limb ($b = R_{\odot}$), the delay is approximately 250 microseconds. The Divergent Gravity model predicts this value with the same precision as General Relativity because both rely on the potential-to-$c^2$ ratio. However, in our framework, this is a direct measurement of flux impedance rather than ``time-time'' metric components.



% -- Start Closing Arguments --
\section{Results: Statistical Validation of the Machian Fit}

The application of the Divergent Gravity model to the SPARC database yields a remarkable convergence between theoretical prediction and observational data. 

\subsection{Standard Error and Convergence}
The primary result of this study is the attainment of 0.5\% precision in the velocity profiles of late-type galaxies. By utilizing the Machian identity for $a_{0}$, we eliminate the need for the ``Dark Matter'' degree of freedom. The residuals between the Divergent Gravity prediction ($V_{dg}$) and the SPARC observations ($V_{obs}$) are statistically consistent with Gaussian noise, indicating that the baryonic distribution is the sole determinant of the gravitational potential gradient.

\subsection{The $\Upsilon$ Distribution}
The optimized mass-to-light ratios ($\Upsilon$) across the sample demonstrate a narrow distribution centered around $0.6 \pm 0.1$. This consistency suggests that the ``Shadowing Efficiency''' is a stable property of baryonic matter, further reinforcing the model's predictive power. Unlike Dark Matter fits, which require varying halo scales for similar galaxies, Divergent Gravity provides a uniform solution governed by the universal background potential $\Phi_{\infty} = c^{2}$.

\subsection{Elimination of the Radial Acceleration Relation (RAR) Scatter}
In standard cosmology, the Radial Acceleration Relation (RAR) exhibits a characteristic scatter that is difficult to explain without fine-tuning halo parameters. In this model, the RAR is a direct consequence of the flux impedance. The 0.5\% precision achieved here suggests that the scatter in the RAR is not intrinsic to the physics of gravity, but is an artifact of failing to account for the isotropic repulsive background.

\section{Conclusion}
The Divergent Gravity model offers a parsimonious resolution to the primary challenges of modern astrophysics. By treating gravity as a displacement of the universal flux rather than a particle-driven attraction, we have successfully derived:
\begin{itemize}
    \item The Gravitational Constant $G = c^{2} R_{U} / M_{U}$.
    \item The cumulative universal mass $M_{U} = \Phi_\infty R_{U} / G$
    \item Our kinematic velocity $v_{0} = c \sqrt{1 - 1 / \left( 1 + g/2R_{\oplus} \right)^{2}}$
    \item The gravitational lens equation that precisely matches that found in GR.
    \item The MOND-regime acceleration $a_{0} = c^{2} / R_{U}$.
    \item The 0.5\% precision fits of galactic rotation curves without non-baryonic halos.
    \item The numerical reconciliation of the Hubble Tension via a local potential shadow of $\approx 0.082 c^{2}$.
\end{itemize}
This framework suggests that the universe is a static Machian system, and that the perceived ``missing mass'' and ``expansion tension'' are geometric consequences of flux impedance and potential gradients.


\begin{table}
    \caption{This table summarizes the predictive accuracy and theoretical parsimony of the two frameworks across multiple scales. In the local regime, DG recovers the successes of GR (Mercury’s precession, gravitational lensing, and Shapiro delay) through a potential shadow gradient. At the galactic and cosmological scales, DG replaces the need for Dark Matter and Dark Energy by deriving the constants $G$ and $a_{0}$ from a single Machian identity: $\Phi_{\infty} = c^{2}$. Notably, the model resolves the $5\sigma$ Hubble Tension with $0.5\%$ precision by accounting for the flux impedance of the Laniakea Supercluster, whereas $\Lambda$CDM requires new physics or remains in tension. The ``Identity'' classification in the DG column signifies that the values are derived mathematical requirements of the model rather than empirical fits.}
    \centering
    \begin{tabular}{l c c c c}
        \hline
            Phenomenon & Standard Model (ΛCDM) & Divergent Gravity & Precision / Match \\
        \hline
            Local Tests \\ (Mercury \&  Lensing) & General Relativity (Metric) & Potential Shadow Impedance & Matches GR exactly  \\
            Rotation Curves & Dark Matter Halos (Fit) & $v4=GMa0$​ (Derived) & $0.5\%$ (SPARC) \\
            Expansion Rate & Dark Energy ($\Lambda$) & Isotropic Repulsive Flux & Matches Planck  \\
            Hubble Tension & Unresolved (5$\sigma$ gap) & Local Potential Shadow & $0.5\%$ (Reconciled)  \\
            Foundational Constants & Arbitrary / Empirical & G, $a_{0}$ derived from $M_{U}$, $R_{U}$ & Mathematical Identity \\
        \hline
    \end{tabular}
    \label{conclusionResultsTable}
\end{table}


% TODO: Add this back in with permission from the journal. Ask to acknowledge those that supported this pursuit while I was homeless
% \ack{Sample text inserted for demonstration.}

\roles{
    Andrew Mueller is an independent researcher and the sole author.
}

\data{Sample text inserted for demonstration.}
% For more information on IOP Publishing's research data policy see: https://publishingsupport.iopscience.iop.org/questions/research-data/

\suppdata{Sample text inserted for demonstration.}


\bibliographystyle{iopart-num}
\bibliography{references}


\end{document}


